\documentclass[]{article}

%Packages
\usepackage{hyperref}
\usepackage[margin=1in]{geometry}
\usepackage{setspace}

\usepackage{graphicx}
\usepackage[usenames, dvipsnames]{color}
\usepackage{setspace}
\usepackage{float}

%Package Settings
\onehalfspacing
\setlength\parindent{0pt}


%opening
\title{DIME Dynamic Documentation Training \\ Stata Exercise}
\author{Luiza Andrade \& Mrijan Rimal}
\date{\today}

\makeatother

\begin{document}

\makeatletter
\begin{titlepage}
	\begin{center}
		\includegraphics[width=0.3\linewidth]{i2i.png}\\[10ex]
		{\LARGE \bfseries  \@title }\\[2ex] 
		{\Large  \@author}\\[20ex] 
		{\large \@date}
	\end{center}
\end{titlepage}
\makeatother
\section*{Read First}
The main things to consider while writing you do-file for dynamic documentation are as follows: 

\begin{itemize}
	\item The format that Stata exports should be readable by the {\LaTeX}. This means that when we export tables it will be exported in \texttt{.tex} format and when we export graphs it should be in \texttt{.png} format. 
		\subitem The files will not be exported in \texttt{MS Word} or \texttt{MS Excel} format.
	
	\item Exported tables and files should be in a folder that is accessible to {\LaTeX}. This means that extra care should be put in the first time you are typing out the path of the directory where the files are put. 
 \end{itemize}
\section*{Setting your own path directories}

\textbf{Path} refers to location of the directory where the files are saved. \\

Before beginning the Stata exercises, we need to set our own paths to the global \texttt{main\_folder} and global \texttt{output} at the start of the do-file. This ensures that the graphs and figures are all saved in a directory {\LaTeX} can access. 

\subsection*{Finding Path on a Windows Computer}
\begin{figure}[H]
	\centering
	\includegraphics[width=1\linewidth]{pathwin}
	\caption{Finding Path on a Windows Computer}
	\label{fig:pathwin}
\end{figure}

As shown in Figure \ref{fig:pathwin}, left clicking(normal click) on the bar at the top of the \texttt{File Explorer} windows where our files are saved shows us the complete path to the files in a Windows computer. \\

\begin{figure}[H]
	\centering
	\includegraphics[width=1\linewidth]{pathwin2}
	\caption{Path shown on a Windows Computer}
	\label{fig:pathwin2}
\end{figure}

We can see in Figure \ref{fig:pathwin2}, that the complete path to the folder is show. We can then use this path when setting the path in our Stata do-file and change the path where it says \begin{verbatim}
global main_folder ``<<<ENTER YOUR FOLDER PATH HERE>>>''
\end{verbatim} 

\subsection*{Finding Path on a Mac}

\section*{Task 1: Manually Create a Graph and then Export it}

This task shows us how to export graphs created in Stata to export in a format that {\LaTeX} can read. Using the \verb|graph export "\$output/regular\_graph.png", width(5000) replace| 
exports the graph in png format which {\LaTeX} can read. 
\section*{Exercise 2, Task 2: Using \texttt{iegraph} to create a figure}

This exercise teaches how to use \texttt{iegraph} to create a figure and export it to the graphs folder. \\

Using the \verb|save(``$output/iegraph.png'')| ensures that the graphs are directly saved to the \texttt{output} folder. 

\section*{}
	
\end{document}
