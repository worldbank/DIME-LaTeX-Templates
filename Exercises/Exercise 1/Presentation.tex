
\documentclass[aspectratio=169]{beamer}

\usetheme{metropolis}           % Use metropolis theme
\usepackage[utf8]{inputenc}
\usepackage{graphicx}
\usepackage{eso-pic}
\usepackage{graphics}
\usepackage{tikz}
\usepackage{verbatim}
\usepackage[export]{adjustbox}
\usepackage{multicol}
\usepackage{listings}
\usepackage{helvet}
\usepackage{booktabs}
\usepackage{threeparttable}
\usepackage{marvosym}
\usepackage{hyperref}
\usepackage{soul}	% For strike-through
\usepackage{tcolorbox} % For color box

\title{Intro to {\LaTeX}}
\date{}
\author{Luiza Andrade} % Name of author(s) of session here
\institute{Development Impact Evaluation (DIME) \newline The World Bank }
\setbeamercolor{background canvas}{bg=white}	% Sets background color

% The below command places the World Bank logo and DIME logo to the right corner
\titlegraphic{%
	\begin{picture}(0,0)
	\put(330,-180){\makebox(0,0)[rt]{\includegraphics[width=3cm]{../img/WB_logo}}}
	\end{picture}%
	\begin{picture}(0,0)
	\put(390,-180){\makebox(0,0)[rt]{\includegraphics[width=1.5cm]{../img/i2i}}}
	\end{picture}%
}

%%% Section page with picture of Light bulb
\makeatletter
\defbeamertemplate*{section page}{mytheme}[1][]{
	\centering
	\begin{minipage}{22em}
		\raggedright
		\usebeamercolor[fg]{section title}
		\usebeamerfont{section title}
		\par
		\ifx\insertsubsectionhead\@empty\else%
		\usebeamercolor[fg]{subsection title}%
		\usebeamerfont{subsection title}%
		\fi
		\ifstrempty{#1}{}{%
			\includegraphics[width=100mm, height=60mm]{#1}%
		}
		\\
		\insertsectionhead\\[-1ex]
		\insertsubsectionhead
		\usebeamertemplate*{progress bar in section page}
		
	\end{minipage}
	\par
	\vspace{\baselineskip}
}
\makeatother

%%% Define a command to include picture in section, 
%%% make section, and revert to old template
\newcommand{\sectionpic}[2]{
	\setbeamertemplate{section page}[mytheme][#2]
	\section{#1}
	\setbeamertemplate{section page}[mytheme]
}

%%% The command below allows for the text that contains Stata code
\lstset{ %
	backgroundcolor=\color{white},
	basicstyle=\tiny,
	breakatwhitespace=false,
	breaklines=true,
	captionpos=b,
	commentstyle=\color{green},
	escapeinside={\%*}{*)},
	extendedchars=true,
	frame=single,
	numbers=left,
	numbersep=5pt,
	numberstyle=\tiny\color{gray},
	rulecolor=\color{black},
	showspaces=false,
	showstringspaces=false,
	showtabs=false,
	stringstyle=\color{mauve},
	tabsize=2,
	title=\lstname,
	morekeywords={not,\},\{,preconditions,effects },
	deletekeywords={time}
}

\definecolor{munsell}{rgb}{0.0, 0.5, 0.69}

%% The below command creates the ligh bulb logos in the top right corner of the 
\begin{document}
	
	
	%%%%%%%%%%%%%%%%%%%%%%%%%%%%%%%%%%%%%%%%%%% Title slide
	{
		\usebackgroundtemplate{\includegraphics[height=55mm,right]{../img/top_right_corner.pdf}} 
		\maketitle
	}

\begin{frame}
	\frametitle{Overview} % Table of contents slide, comment this block out to remove it
	\tableofcontents % Throughout your presentation, if you choose to use \section{} and \subsection{} commands, these will automatically be printed on this slide as an overview of your presentation
\end{frame}

%%%%%%%%%%%%%%%%%%%%%%%%%%%%%%%%%%%%%%%%%%% Section title slide
\sectionpic{Introduction}{../img/section_slide}

%%%%%%%%%%%%%%%%%%%%%%%%%%%%%%%%%%%%%%%%%%% Regular slides
\begin{frame}{Introduction}
	\begin{itemize}
		\item This session will introduce you to some basic concepts of {\LaTeX}
		\item {\LaTeX} documents will be compiled with the latest version of the files it uses as inputs
		\item At the end of this session, you will have a template document that incorporates updates to tables and figures exported from statistical software without the need to copy and paste them
		\item The document we will create is very simple, as the full features of {\LaTeX} cannot be explored in an hour and a half
		\item If you want to have a more broad understanding of {\LaTeX} or explore other functionalities, the \textcolor{blue}{\href{https://en.wikibooks.org/wiki/LaTeX}{{\LaTeX} Wikibook}} is a great resource. 
	\end{itemize}
\end{frame}

\begin{frame}{Get started}
	Using a desktop version of \LaTeX:
	\begin{enumerate}
		\item Make sure you have both {\LaTeX} and a {\LaTeX} editor installed in your computer. Installation instructions can be found \textcolor{blue}{\href{https://github.com/worldbank/DIME-LaTeX-Templates/blob/master/Set\%20up\%20LaTeX\%20on\%20your\%20computer/How\%20to\%20set\%20up\%20TeXstudio.pdf}{here}}
		\item Go to \url{https://osf.io/8sgrh/files/}
		\item Click on \textit{2020\_12 - RA Continuing Education ({\LaTeX})} and then \texttt{Download as zip}
		\item Unzip the \textit{2020\_12 - RA Continuing Education ({\LaTeX})} folder
		\item Navigate to the folder \textit{2020\_12 - RA Continuing Education ({\LaTeX})/Output/Final} and open the file \textit{appendix.tex}
	\end{enumerate}
\end{frame}

\begin{frame}{Get started}
	Using Overleaf:
	\begin{enumerate}
		\item Create an account at \url{https://www.overleaf.com}
		\item Go to \url{https://www.overleaf.com/read/zfgkzgfkybxs}
		\item Click on the arrow next to \emph{Menu} on the top left corner to see your projects
		\item On the rightmost column of the line for the project \textit{Intro to \LaTeX}, click on the \textit{Copy} icon
		\item Open the project \textit{Intro to \LaTeX(Copy)} and navigate to \textit{Final/appendix.tex} under the file-tree on the left menu
	\end{enumerate}
\end{frame}

%%%%%%%%%%%%%%%%%%%%%%%%%%%%%%%%%%%%%%%%%%% Section title slide
\sectionpic{Folders and relative file paths}{../img/section_slide}

\begin{frame}{Folders and relative file paths}
	\begin{tcolorbox}[colback=white,colframe=orange,title=\textbf{Important!}]
		{\LaTeX} uses relative file paths, so to import any graphs or figures to \textit{appendix.tex}, your path reference starts in the folder where it is saved.
	\end{tcolorbox}
\end{frame}

\begin{frame}{Folders and relative file paths}
	\begin{itemize}
		\item An absolute or full path points to the same location in a file system, regardless of the current working directory. To do that, it must include the root directory.
		\item A relative path, on the other hand, starts from some given working directory, avoiding the need to provide the full absolute path.
		\item For any given {\LaTeX} document, we must use relative file paths starting in the same directory where that document is save.
		\item For this session purposes, all file paths will start in the folder where \textit{appendix.tex} is.
		\item From this location, we can go to a parent directory using \texttt{..} or to a child directory using \texttt{child-directory-name/}
		\item The relative file path from \textit{appendix.tex} to, for example, image \textit{iegraph.png} in the \textit{Raw} folder is \textit{../Raw/iegraph.png}
	\end{itemize}
\end{frame}

%%%%%%%%%%%%%%%%%%%%%%%%%%%%%%%%%%%%%%%%%%% Section title slide
\sectionpic{Compiling a PDF file from  {\LaTeX} code}{../img/section_slide}

\begin{frame}{Compiling a pdf file from  {\LaTeX} code}
	\begin{tcolorbox}[colback=white,colframe=orange,title=\textbf{Important!}]
		The \textit{.tex} file is not your final document. The \textit{.tex} file is used to create your final document. For that, you need to compile your code. The keyboard shortcut to that is \textit{F5}.
	\end{tcolorbox}
\end{frame}

\begin{frame}{Compiling a pdf file from  {\LaTeX} code}
\begin{itemize}
	\item ``Compiling'' means that you run your code to create and save a PDF version of your document
	\item It is similar to running a do-file in Stata or an R-script in R, but it creates a PDF instead of running computations. 
\end{itemize}
\end{frame}

\begin{frame}{Compiling a pdf file from  {\LaTeX} code}
\begin{tcolorbox}[colback=white,colframe=munsell,title=\textbf{Exercise}]
	\begin{enumerate}
		\item Open the file \textit{appendix.tex}
		\item To compile it in TeXstudio, click the green double forward arrow or hit \texttt{F5}. 
				The pdf file was saved in the same folder as where the file \textit{appendix.tex} is saved. {\LaTeX} exports a lot of auxiliary files as well here. You do not need these files.
		\item To compile it in Overleaf, click the \textit{Recompile} button. The compiled file will show on the right side of the screen
	\end{enumerate} 
\end{tcolorbox}
\end{frame}

\begin{frame}[fragile]{Compiling a pdf file from  {\LaTeX} code}
	\begin{itemize}
		\item Note that the content included in the PDF is what is shown between \verb|\begin{document}| and \verb|\end{document}|
		\item The preamble at the top of the document loads different functionalities, but how to use these settings is beyond the scope of this session
	\end{itemize} 
\end{frame}

%%%%%%%%%%%%%%%%%%%%%%%%%%%%%%%%%%%%%%%%%%% Section title slide
\sectionpic{Using TeXstudio interface to import an image into \LaTeX}{../img/section_slide}

\begin{frame}{Using TeXstudio interface to import an image into \LaTeX}
\begin{tcolorbox}[colback=white,colframe=orange,title=\textbf{Take-away}]
	The easiest way to import a figure to {\LaTeX} is to simply drag and drop it. This is also an easy way to find out the relative path to your file. However, drag and drop is a feature of TeXstudio, and may not work in other editors, including Overleaf.
\end{tcolorbox}
\end{frame}

\begin{frame}[fragile]{Using TeXstudio interface to import an image into \LaTeX}
\begin{itemize}
	\item Note that the content included in the PDF is what is shown between \verb|\begin{document}| and \verb|\end{document}|
	\item The preamble at the top of the document loads different functionalities, but how to use these settings is beyond the scope of this session
\end{itemize} 
\end{frame}

\begin{frame}{Using TeXstudio interface to import an image into \LaTeX}
\begin{tcolorbox}[colback=white,colframe=munsell,title=\textbf{Exercise}]
	\begin{enumerate}
		\item While you keep \textit{appendix.tex} open in TeXstudio, open the \textit{Raw} folder in regular file explorer
		\item Now drag and drop figure \textit{take\_up.png} it to the part of \textit{appendix.tex}  that says ``This is a blank document''
		\item Click ``OK'' in the window that pops up
		\item Compile your document again
	\end{enumerate} 
\end{tcolorbox}
\end{frame}

\begin{frame}[fragile]{Using TeXstudio interface to import an image into \LaTeX}
The previous exercise will create a section of code that should look similar to the code below. 
\begin{center}
	\begin{verbatim}
	\begin{figure}
	\centering
	\includegraphics[width=0.7\linewidth]{../Raw/take_up}
	\caption{}
	\label{fig:take_up}
	\end{figure}
	\end{verbatim}
\end{center}
\end{frame}

\begin{frame}[fragile]{Using TeXstudio interface to import an image into \LaTeX}
	Note that the relative file path to your \emph{Raw} folder was created in end of this line:
\begin{center}
	\begin{verbatim}
	\includegraphics[width=0.7\linewidth]{../Raw/take_up}
	\end{verbatim}
\end{center}
\end{frame}

\begin{frame}[fragile]{Using TeXstudio interface to import an image into \LaTeX}
\begin{itemize}
	\item You will see \verb|\begin{}| and \verb|\end{}| a lot in {\LaTeX}
	\item The define where one object start and end, and the lines of code in between \verb|\begin{}| and \verb|\end{}| is called an environment -- in this case, a figure environment
	\item These delimiters tell {\LaTeX} that you are importing a figure and any other code in that environment in addition to the line with the relative path to the figure is a setting or provide some meta information to that figure.
\end{itemize}
\end{frame}

%%%%%%%%%%%%%%%%%%%%%%%%%%%%%%%%%%%%%%%%%%% Section title slide
\sectionpic{Using code to import an image into \LaTeX}{../img/section_slide}

\begin{frame}[fragile]{Using code to import an image into \LaTeX}
\begin{tcolorbox}[colback=white,colframe=munsell,title=\textbf{Exercise}]
	Below the code that was generated when you imported the first figure (but before \verb|\end{document}|), create a figure environment and import \emph{iegraph.png} into it. You will find that graph in the \emph{Raw} folder. You import that figure by typing the following lines of code:
	\begin{center}
		\begin{verbatim}
		\begin{figure}[H]
		\includegraphics{../Raw/iegraph.png}
		\end{figure}
		\end{verbatim}
	\end{center}
	Remember to include \verb|../Raw/| as this is the relative path to the \emph{Raw} folder where the graph is saved. Compile your document. How does it look? 
\end{tcolorbox}
\end{frame}

\begin{frame}[fragile]{Using code to import an image into \LaTeX}
	\begin{figure}[H]
	\includegraphics{Output/Raw/iegraph.png}
	\end{figure}
\end{frame}

\begin{frame}[fragile]{Using code to import an image into \LaTeX}
\begin{tcolorbox}[colback=white,colframe=munsell,title=\textbf{Exercise}]
	Adjust the size of your figure using the \verb|\width| option of \verb|\includegraphics{}|:
	\begin{center}
		\begin{verbatim}
		\begin{figure}[H]
		\includegraphics[width=\textwidth]{../Raw/iegraph.png}
		\end{figure}
		\end{verbatim}
	\end{center}
\end{tcolorbox}

The property \verb|=\textwidth| means that the figure will be as wide as the margins of the text in your document

\end{frame}

\begin{frame}[fragile]{Using code to import an image into \LaTeX}
\begin{tcolorbox}[colback=white,colframe=munsell,title=\textbf{Exercise}]
	Add a title to your figure using the \verb|\caption| command:
	\begin{center}
		\begin{verbatim}
		\begin{figure}[H]
		\includegraphics[width=\textwidth]{../Raw/iegraph.png}
		\caption{Add figure title here}
		\end{figure}
		\end{verbatim}
	\end{center}

Change the caption from \verb|{Add figure title here}| to any title of your choice and compile the document again.
\end{tcolorbox}
\end{frame}

%%%%%%%%%%%%%%%%%%%%%%%%%%%%%%%%%%%%%%%%%%% Section title slide
\sectionpic{Using code to import an image into \LaTeX}{../img/section_slide}

\begin{frame}[fragile]{ Importing tables into \LaTeX}
\begin{tcolorbox}[colback=white,colframe=munsell,title=\textbf{Exercise}]
	\begin{enumerate}
		\item Create a table environment by using \verb|\begin{}| and \verb|\end{}| underneath the two figures you already imported
		\begin{itemize}
			 \item Instead of writing \verb|figure| in \verb|\begin{}| and \verb|\end{}| as you did above, write \verb|table| 
			 \item This tells {\LaTeX} to expect a table instead if a figure
		\end{itemize}
		\item Import \emph{regression\_table.tex} into the table environment using \verb|\input{}|.
		\begin{itemize}
			\item Inside input, add the relative file path to the table
		\end{itemize}
	\end{enumerate}
\end{tcolorbox}
\end{frame}

\begin{frame}[fragile]{ Importing tables into \LaTeX}

	The code should look like this:
	
	\begin{center}
	\begin{verbatim}
	\begin{table}[H]
	{
\def\sym#1{\ifmmode^{#1}\else\(^{#1}\)\fi}
\begin{tabular}{l*{8}{c}}
\hline\hline
                    &\multicolumn{1}{c}{(1)}&\multicolumn{1}{c}{(2)}&\multicolumn{1}{c}{(3)}&\multicolumn{1}{c}{(4)}&\multicolumn{1}{c}{(5)}&\multicolumn{1}{c}{(6)}&\multicolumn{1}{c}{(7)}&\multicolumn{1}{c}{(8)}\\
                    
\hline
Post        &      -0.304\sym{*}  &      -0.632\sym{***}&       0.003         &       0.217\sym{**} &      -0.059         &      -0.012         &       0.136         &       0.333         \\
                    &     (0.141)         &     (0.026)         &     (0.130)         &     (0.072)         &     (0.082)         &     (0.029)         &     (0.365)         &     (0.265)         \\
[1em]
Treatment     &      -0.055         &       0.267\sym{***}&      -0.133\sym{*}  &       0.096         &      -0.210\sym{*}  &      -0.035         &      -0.117         &      -0.033\sym{***}\\
                    &     (0.152)         &     (0.000)         &     (0.064)         &     (0.105)         &     (0.098)         &     (0.078)         &     (0.133)         &     (0.000)         \\
[1em]
Post * treatment &               &               &       0.065         &      -0.254\sym{*}  &       0.111         &      -0.000         &       0.284         &      -0.585         \\
                    &                 &                &     (0.159)         &     (0.116)         &     (0.086)         &     (0.033)         &     (0.378)         &     (0.296)         \\
[1em]
Constant            &       0.581\sym{***}&       0.333\sym{***}&       0.378\sym{***}&       0.175         &       0.276\sym{**} &       0.039         &       0.364\sym{**} &       0.000         \\
                    &     (0.131)         &     (0.000)         &     (0.055)         &     (0.105)         &     (0.095)         &     (0.081)         &     (0.115)         &     (0.000)         \\
\hline
Observations        &         354         &         354         &        1897         &        1897         &        1652         &        1652         &         348         &         348         \\
Fixed-effects  &          No         &         Yes         &          No         &         Yes         &          No         &         Yes         &          No         &         Yes         \\

\(R^{2}\)           &       0.033         &       0.595         &       0.014         &       0.287         &       0.043         &       0.386         &       0.076         &       0.535         \\

\hline\hline
\multicolumn{9}{l}{\footnotesize Standard errors clustered at user level are in parentheses. \sym{*} \(p<0.05\), \sym{**} \(p<0.01\), \sym{***} \(p<0.001\)}\\
\end{tabular}
}

	\end{table}
	\end{verbatim}
	\end{center}

	Now compile the document and look at the table. 
\end{frame}

\begin{frame}[fragile]{ Importing tables into \LaTeX}

\begin{center}
	\begin{table}[H]
	{
\def\sym#1{\ifmmode^{#1}\else\(^{#1}\)\fi}
\begin{tabular}{l*{8}{c}}
\hline\hline
                    &\multicolumn{1}{c}{(1)}&\multicolumn{1}{c}{(2)}&\multicolumn{1}{c}{(3)}&\multicolumn{1}{c}{(4)}&\multicolumn{1}{c}{(5)}&\multicolumn{1}{c}{(6)}&\multicolumn{1}{c}{(7)}&\multicolumn{1}{c}{(8)}\\
                    
\hline
Post        &      -0.304\sym{*}  &      -0.632\sym{***}&       0.003         &       0.217\sym{**} &      -0.059         &      -0.012         &       0.136         &       0.333         \\
                    &     (0.141)         &     (0.026)         &     (0.130)         &     (0.072)         &     (0.082)         &     (0.029)         &     (0.365)         &     (0.265)         \\
[1em]
Treatment     &      -0.055         &       0.267\sym{***}&      -0.133\sym{*}  &       0.096         &      -0.210\sym{*}  &      -0.035         &      -0.117         &      -0.033\sym{***}\\
                    &     (0.152)         &     (0.000)         &     (0.064)         &     (0.105)         &     (0.098)         &     (0.078)         &     (0.133)         &     (0.000)         \\
[1em]
Post * treatment &               &               &       0.065         &      -0.254\sym{*}  &       0.111         &      -0.000         &       0.284         &      -0.585         \\
                    &                 &                &     (0.159)         &     (0.116)         &     (0.086)         &     (0.033)         &     (0.378)         &     (0.296)         \\
[1em]
Constant            &       0.581\sym{***}&       0.333\sym{***}&       0.378\sym{***}&       0.175         &       0.276\sym{**} &       0.039         &       0.364\sym{**} &       0.000         \\
                    &     (0.131)         &     (0.000)         &     (0.055)         &     (0.105)         &     (0.095)         &     (0.081)         &     (0.115)         &     (0.000)         \\
\hline
Observations        &         354         &         354         &        1897         &        1897         &        1652         &        1652         &         348         &         348         \\
Fixed-effects  &          No         &         Yes         &          No         &         Yes         &          No         &         Yes         &          No         &         Yes         \\

\(R^{2}\)           &       0.033         &       0.595         &       0.014         &       0.287         &       0.043         &       0.386         &       0.076         &       0.535         \\

\hline\hline
\multicolumn{9}{l}{\footnotesize Standard errors clustered at user level are in parentheses. \sym{*} \(p<0.05\), \sym{**} \(p<0.01\), \sym{***} \(p<0.001\)}\\
\end{tabular}
}

	\end{table}
\end{center}
\end{frame}

\begin{frame}[fragile]{Importing tables into \LaTeX}
\begin{itemize}
	\item To adjust the table size, we will use the \verb|\adjustbox| environment
	\item This environment can be created \textbf{inside a table environment} using the following code:
	\begin{verbatim}
	\begin{table}[H]
	\begin{adjustbox}{max width=\textwidth}
	\input{}
	\end{adjustbox}
	\end{table}
	\end{verbatim}
	\item All the content added between \verb|\begin{adjustbox}| and \verb|\end{adjustbox}| will have the property \verb|{max width=\textwidth}|.
\end{itemize}
\end{frame}

\begin{frame}[fragile]{Importing tables into \LaTeX}
\begin{tcolorbox}[colback=white,colframe=munsell,title=\textbf{Exercise}]
	Edit the code to import your table so the \verb|\input{}| line is inside an \verb|\end{adjustbox}| environment
\end{tcolorbox}

\begin{center}
	\begin{verbatim}
	\begin{table}[H]
	\begin{adjustbox}{max width=\textwidth}
	{
\def\sym#1{\ifmmode^{#1}\else\(^{#1}\)\fi}
\begin{tabular}{l*{8}{c}}
\hline\hline
                    &\multicolumn{1}{c}{(1)}&\multicolumn{1}{c}{(2)}&\multicolumn{1}{c}{(3)}&\multicolumn{1}{c}{(4)}&\multicolumn{1}{c}{(5)}&\multicolumn{1}{c}{(6)}&\multicolumn{1}{c}{(7)}&\multicolumn{1}{c}{(8)}\\
                    
\hline
Post        &      -0.304\sym{*}  &      -0.632\sym{***}&       0.003         &       0.217\sym{**} &      -0.059         &      -0.012         &       0.136         &       0.333         \\
                    &     (0.141)         &     (0.026)         &     (0.130)         &     (0.072)         &     (0.082)         &     (0.029)         &     (0.365)         &     (0.265)         \\
[1em]
Treatment     &      -0.055         &       0.267\sym{***}&      -0.133\sym{*}  &       0.096         &      -0.210\sym{*}  &      -0.035         &      -0.117         &      -0.033\sym{***}\\
                    &     (0.152)         &     (0.000)         &     (0.064)         &     (0.105)         &     (0.098)         &     (0.078)         &     (0.133)         &     (0.000)         \\
[1em]
Post * treatment &               &               &       0.065         &      -0.254\sym{*}  &       0.111         &      -0.000         &       0.284         &      -0.585         \\
                    &                 &                &     (0.159)         &     (0.116)         &     (0.086)         &     (0.033)         &     (0.378)         &     (0.296)         \\
[1em]
Constant            &       0.581\sym{***}&       0.333\sym{***}&       0.378\sym{***}&       0.175         &       0.276\sym{**} &       0.039         &       0.364\sym{**} &       0.000         \\
                    &     (0.131)         &     (0.000)         &     (0.055)         &     (0.105)         &     (0.095)         &     (0.081)         &     (0.115)         &     (0.000)         \\
\hline
Observations        &         354         &         354         &        1897         &        1897         &        1652         &        1652         &         348         &         348         \\
Fixed-effects  &          No         &         Yes         &          No         &         Yes         &          No         &         Yes         &          No         &         Yes         \\

\(R^{2}\)           &       0.033         &       0.595         &       0.014         &       0.287         &       0.043         &       0.386         &       0.076         &       0.535         \\

\hline\hline
\multicolumn{9}{l}{\footnotesize Standard errors clustered at user level are in parentheses. \sym{*} \(p<0.05\), \sym{**} \(p<0.01\), \sym{***} \(p<0.001\)}\\
\end{tabular}
}

	\end{adjustbox}
	\end{table}
	\end{verbatim}
\end{center}
\end{frame}

\begin{frame}[fragile]{Importing tables into \LaTeX}
	\begin{table}[H]
	\begin{adjustbox}{max width=\textwidth}
	{
\def\sym#1{\ifmmode^{#1}\else\(^{#1}\)\fi}
\begin{tabular}{l*{8}{c}}
\hline\hline
                    &\multicolumn{1}{c}{(1)}&\multicolumn{1}{c}{(2)}&\multicolumn{1}{c}{(3)}&\multicolumn{1}{c}{(4)}&\multicolumn{1}{c}{(5)}&\multicolumn{1}{c}{(6)}&\multicolumn{1}{c}{(7)}&\multicolumn{1}{c}{(8)}\\
                    
\hline
Post        &      -0.304\sym{*}  &      -0.632\sym{***}&       0.003         &       0.217\sym{**} &      -0.059         &      -0.012         &       0.136         &       0.333         \\
                    &     (0.141)         &     (0.026)         &     (0.130)         &     (0.072)         &     (0.082)         &     (0.029)         &     (0.365)         &     (0.265)         \\
[1em]
Treatment     &      -0.055         &       0.267\sym{***}&      -0.133\sym{*}  &       0.096         &      -0.210\sym{*}  &      -0.035         &      -0.117         &      -0.033\sym{***}\\
                    &     (0.152)         &     (0.000)         &     (0.064)         &     (0.105)         &     (0.098)         &     (0.078)         &     (0.133)         &     (0.000)         \\
[1em]
Post * treatment &               &               &       0.065         &      -0.254\sym{*}  &       0.111         &      -0.000         &       0.284         &      -0.585         \\
                    &                 &                &     (0.159)         &     (0.116)         &     (0.086)         &     (0.033)         &     (0.378)         &     (0.296)         \\
[1em]
Constant            &       0.581\sym{***}&       0.333\sym{***}&       0.378\sym{***}&       0.175         &       0.276\sym{**} &       0.039         &       0.364\sym{**} &       0.000         \\
                    &     (0.131)         &     (0.000)         &     (0.055)         &     (0.105)         &     (0.095)         &     (0.081)         &     (0.115)         &     (0.000)         \\
\hline
Observations        &         354         &         354         &        1897         &        1897         &        1652         &        1652         &         348         &         348         \\
Fixed-effects  &          No         &         Yes         &          No         &         Yes         &          No         &         Yes         &          No         &         Yes         \\

\(R^{2}\)           &       0.033         &       0.595         &       0.014         &       0.287         &       0.043         &       0.386         &       0.076         &       0.535         \\

\hline\hline
\multicolumn{9}{l}{\footnotesize Standard errors clustered at user level are in parentheses. \sym{*} \(p<0.05\), \sym{**} \(p<0.01\), \sym{***} \(p<0.001\)}\\
\end{tabular}
}

	\end{adjustbox}
	\end{table}
\end{frame}

\begin{frame}[fragile]{Importing tables into \LaTeX}
\begin{itemize}
	\item Similar to figures, you can add a caption to your table. 
	\item This is done by adding the same lines as for figures. 
	\item Note that the caption needs to be outside of the \texttt{adjustbox} environment, but inside the \texttt{table} environment.
\end{itemize}
\end{frame}

\begin{frame}[fragile]{Importing tables into \LaTeX}
\begin{tcolorbox}[colback=white,colframe=munsell,title=\textbf{Exercise}]
	\begin{enumerate}
		\item Update the text in \verb|\caption{Add figure title here}| and compile the document
		\begin{itemize}
			\item Note the title of the table is above the table
			\item When we used \verb|\caption{}| for the figure earlier, it was below 
			\item That has nothing to do with tables and figures, it all depends if you enter the line with \verb|\caption{}| before or after the line when you are importing a file to your document. 
		\end{itemize}
		\item Move the line with the caption to after \verb|\end{adjustbox}| and before \verb|\end{table}| and compile the document again. 
		\begin{itemize}
			\item The caption line must come before the \texttt{adjustbox} environment begins, otherwise the document will not compile.
		\end{itemize}
	\end{enumerate}
\end{tcolorbox}
\end{frame}


%%%%%%%%%%%%%%%%%%%%%%%%%%%%%%%%%%%%%%%%%%% Section title slide
\sectionpic{Centering imported objects}{../img/section_slide}

\begin{frame}[fragile]{Centering imported objects}
\begin{itemize}
	\item Until now, we imported object into our document that were too wide for our document's margins, so we had to adjust their sizes. 
	\item Next, we are going to see what happens when we import a narrower object. 
\end{itemize}	
\begin{tcolorbox}[colback=white,colframe=munsell,title=\textbf{Exercise}]
	\begin{enumerate}
		\item Create a new table environment (\verb|\begin{table}| and \verb|\end{table}|)
		\item Use \verb|input{}| to import the table \textit{sample\_sizes.tex} into the document
		\item Add a title to your table
	\end{enumerate}
\end{tcolorbox}
\end{frame}

\begin{frame}[fragile]{Centering imported objects}
\begin{center}
	\begin{verbatim}
	\begin{table}[H]
	\caption{Add a title to this table}
	{
\def\sym#1{\ifmmode^{#1}\else\(^{#1}\)\fi}
\begin{tabular}{l*{4}{c}}
\hline\hline
          &\multicolumn{1}{c}{Group 1}&\multicolumn{1}{c}{Group 2}&\multicolumn{1}{c}{Group 3}&\multicolumn{1}{c}{Group 4}\\
\hline
Control   &       203         &       945         &        700         &        200         \\
Treatment &       151         &       952         &        952         &        148         \\
\hline
Total     &       354         &       1897         &       1652         &       348         \\
\hline\hline
\end{tabular}
}

	\end{table}
	\end{verbatim}
\end{center}
\end{frame}

\begin{frame}[fragile]{Centering imported objects}
As you can see, this table is not yet centered: 

\begin{center}
	\begin{table}[H]
	\caption{Add a title to this table}
	{
\def\sym#1{\ifmmode^{#1}\else\(^{#1}\)\fi}
\begin{tabular}{l*{4}{c}}
\hline\hline
          &\multicolumn{1}{c}{Group 1}&\multicolumn{1}{c}{Group 2}&\multicolumn{1}{c}{Group 3}&\multicolumn{1}{c}{Group 4}\\
\hline
Control   &       203         &       945         &        700         &        200         \\
Treatment &       151         &       952         &        952         &        148         \\
\hline
Total     &       354         &       1897         &       1652         &       348         \\
\hline\hline
\end{tabular}
}

	\end{table}
\end{center}
\end{frame}

\begin{frame}[fragile]{Centering imported objects}
\begin{tcolorbox}[colback=white,colframe=munsell,title=\textbf{Exercise}]
	Center the table by writing \verb|\centering| inside the \texttt{table} environment.
\end{tcolorbox}

\begin{itemize}
	\item Adding \verb|\centering| to the beginning of an environment centers whatever is in that environment.
	\item That means that if the same line was added after a \verb|\begin{figure}|, that figure would also be centered.
\end{itemize}
\end{frame}

\begin{frame}[fragile]{Centering imported objects}
\begin{center}
	\begin{verbatim}
	\begin{table}[H]
	\centering
	\caption{Add a title to this table}
	{
\def\sym#1{\ifmmode^{#1}\else\(^{#1}\)\fi}
\begin{tabular}{l*{4}{c}}
\hline\hline
          &\multicolumn{1}{c}{Group 1}&\multicolumn{1}{c}{Group 2}&\multicolumn{1}{c}{Group 3}&\multicolumn{1}{c}{Group 4}\\
\hline
Control   &       203         &       945         &        700         &        200         \\
Treatment &       151         &       952         &        952         &        148         \\
\hline
Total     &       354         &       1897         &       1652         &       348         \\
\hline\hline
\end{tabular}
}

	\end{table}
	\end{verbatim}
\end{center}
\end{frame}


%%%%%%%%%%%%%%%%%%%%%%%%%%%%%%%%%%%%%%%%%%% Section title slide
\sectionpic{Adding a document title}{../img/section_slide}

\begin{frame}[fragile]{Adding a title to your document}
\begin{itemize}
	\item To add a title to your document, add \verb|\maketitle| on a new line immediately below \verb|\begin{document}|
	\item Compile the document
	\item You can see the information we already added
\end{itemize}
\end{frame}

\begin{frame}[fragile]{Importing tables into \LaTeX}
\begin{tcolorbox}[colback=white,colframe=munsell,title=\textbf{Exercise}]
	Update the title information with a title of your own and with you name in the preamble
	\begin{enumerate}
		\item Go to your document's preamble and look for the \verb|\title{}| information
		\item Change the title to one of your choosing
		\item Find the \verb|\author{}| information and update it as well
		\item Compile the document and see your changes
	\end{enumerate}
\end{tcolorbox}
\end{frame}

%%%%%%%%%%%%%%%%%%%%%%%%%%%%%%%%%%%%%%%%%%% Section title slide
\sectionpic{Adding a list of tables and a list of figures}{../img/section_slide}

\begin{frame}[fragile]{Adding a list of tables and a list of figures}
\begin{tcolorbox}[colback=white,colframe=munsell,title=\textbf{Exercise}]
	Immediately after where you added \verb|\maketitle|, on a new line, add \verb|\listoftables| and, on a second new line, add \verb|\listoffigures|.
\end{tcolorbox}

\begin{center}
	\begin{verbatim}
	\begin{document}
	\maketitle
	\listoftables
	\listoffigures
	\end{verbatim}
\end{center}
\end{frame}

\begin{frame}[fragile]{Adding a list of tables and a list of figures}
\begin{itemize}
	\item Compile your document
	\item The titles shown in these lists are those defined by the \verb|\caption| you added to each table and figure
	\item You can go to any of your figures or tables, change the caption and compile the document again to see how the list of tables and figures is updated
	\item While it is common practice, you do not need to add \verb|\maketitle| and \verb|\listoftables| in the beginning of the document. You can add them anywhere.
\end{itemize}
\end{frame}

%%%%%%%%%%%%%%%%%%%%%%%%%%%%%%%%%%%%%%%%%%% Section title slide
\sectionpic{Useful resources}{../img/section_slide}

\begin{frame}[fragile]{Useful resources}
\begin{itemize}
	\item LaTeX Wikibook: \url{https://en.wikibooks.org/wiki/LaTeX }
	\item More LaTeX tutorials: \url{https://github.com/worldbank/DIME-LaTeX-Templates }
	\item Exporting tables from Stata to LaTeX: \url{https://github.com/worldbank/stata-tables}
	\item Packages to export tables from R to LaTeX: \textcolor{blue}{\href{https://cran.r-project.org/web/packages/stargazer/vignettes/stargazer.pdf}{stargazer}} and \textcolor{blue}{\href{https://cran.r-project.org/web/packages/huxtable/vignettes/huxtable.html}{huxtable}}
\end{itemize}
\end{frame}

\end{document}
