\documentclass[12pts]{article}

\usepackage[margin=1in]{geometry}
\usepackage[usenames, dvipsnames]{color}
\usepackage{setspace}
\usepackage{graphicx}
\doublespacing
\usepackage{fancyvrb}
\usepackage{varwidth}
\usepackage{verbatim}
\usepackage{multicol}
\usepackage{enumitem}
\usepackage{float}
\usepackage{fancyhdr}
\usepackage{parskip}    % This packages sets the spacing between two paragraphsc   
\usepackage{hyperref}              
\setlength{\parskip}{.5\baselineskip}   % Define spacing between two paragraphs

% Bibliography packages
\usepackage[]{natbib}
\bibliographystyle{chicago}
\usepackage{bibentry} 

\setlength{\parindent}{0pt}
\setlength{\columnseprule}{0pt}

% Title Page
\title{DIME DYNAMIC DOCUMENTATION TRAINING \\ Exercise 3}
\author{Luiza Andrade} 
\date{\today}

\makeatother


\begin{document}
	
	
	\makeatletter
	\begin{titlepage}
		\begin{center}
			\includegraphics[width=0.3\linewidth]{../img/i2i.png}\\[10ex]
			{\LARGE \bfseries  \@title }\\[2ex] 
			{\Large  \@author}\\[20ex] 
			{\large \@date}
		\end{center}
	\textcolor{red}{For the most recent version of the file, please check \url{https://github.com/worldbank/DIME-LaTeX-Templates/}}
	\end{titlepage}
	\makeatother
	
	\tableofcontents
	
	\newpage			
	\section*{Introduction}
	
On Exercises 1 and 2, you learned to use {\LaTeX} to import tables and graphs. This will allow you to create an easily updated file with descriptive statistics and analysis results. However, {\LaTeX} has plenty of other features you can explore to create different types of documents. 

In this exercise, you will explore some intermediate-level tools for text editing and formatting. You will also learn a little bit more about how {\LaTeX} works. Once you're done with it, you should have enough knowledge of {\LaTeX} to write a full paper or report.
	
	
	\section{Text Formatting}
	
	Now that you have created sections in your document, and added text, you might want to change some of the formatting of your text. In {\LaTeX} you format your text using packages and code. We will here include a few topics.
	
	\subsection{Bold, italic and underlined}
	
	You can make part of your text bold using \verb|\textbf{text}|. To 	underline part of your text, use \verb|\underline{text}|. To make part of your text italic use \verb|\textit{text}|. Here's how it looks:
	
	\begin{center}
		\begin{multicols}{2}
			\verb|\textbf{This is a bold text.}| \\
			\verb|\underline{This is an underlined text.}| \\
			\verb|\textit{This is an italic text.}| \\
			
			\columnbreak
			
			\textbf{This is a bold text.} \\
			\underline{This is an underlined text.} \\
			\textit{This is an italic text.}
		\end{multicols}
	\end{center}
	
	\textcolor{BurntOrange}{\textbf{Task 6:}} Choose some words in your text paragraph to be turned into bold, italic and underline. If you're using TeXStudio, you can use the shortcuts \texttt{CTRL+B} for bold and \texttt{CTRL+I} for italic.
	
	\subsection{Text color}
	Font color can be changed using \verb|\textcolor{color}{text}|. You can find a list of color options in the \href{https://en.wikibooks.org/wiki/LaTeX/Colors}{{\LaTeX} WikiBook}, but here are a few examples:
	
	% TO-DO: say that you need to type usepackage in the preamble
	
	\clearpage
	\begin{center}
		\begin{multicols}{2}
			\verb|\textcolor{red}{This is a read text.}| \\
			\verb|\textcolor{Gray}{This is a gray text.}| \\
			\verb|\textcolor{Cyan}{This is a Cyan text.}| \\
			\verb|\textcolor{RoyalPurple}{This is a RoyalPurple text.}|
			
			\columnbreak
			
			\textcolor{red}{This is a read text.} \\
			\textcolor{Gray}{This is a gray text.} \\
			\textcolor{Cyan}{This is a Cyan text.} \\
			\textcolor{RoyalPurple}{This is a RoyalPurple text.}
		\end{multicols}
	\end{center}
	
	\textcolor{BurntOrange}{\textbf{Task 7:}} Change the color of a sentence in your paragraph. Try combining \verb|\textcolor| with \verb|textbf|.
	
	\section{Line Spacing}
	It is common that different publication standards require different line spacing. This can be achieved using the \texttt{setspace} package. 
	
	To use this package you need to start by importing it. To import the package you add \verb|\usepackage{setspace}| next to where a lot of other packages are imported using the  \verb|\usepackage{}| command. Next you simply tell {\LaTeX} which line spacing you want and you need to do this before \verb|begin{document}|. \verb|\singlespacing| makes {\LaTeX} use single line space throughout the document. Similarly, \verb|\doublespacing| and \verb|\onehalfspacing| set the document to double line space and one half line space respectively. See example below:
	
	\begin{Verbatim}[commandchars=+\(\)]
	+color(CornflowerBlue)\usepackage{setspace}
	+color(CornflowerBlue)\doublespacing
	
	\begin{document}
	\end{Verbatim}
	
	\section{Creating lists}
	
	As you have probably noticed in this document, it is also possible to make lists. The \texttt{enumerate} environment created numbered lists, while the \texttt{itemize} environment only creates items.
	
	To make a list, you must first create the environment by typing \verb|\begin{itemize}| and \verb|\end{itemize}|, then type \verb|\item| before each of the items in your list, as shown below:
	
	\begin{multicols}{2}
		\begin{Verbatim}
		Here's a numbered list:
		\begin{enumerate}
		\item Item one
		\item Item two
		\item Item three
		\end{enumerate}
		\end{Verbatim}
		
		\columnbreak	
		
		Here's a numbered list:
		\begin{enumerate}
			\item Item one
			\item Item two
			\item Item three
		\end{enumerate}
	\end{multicols}
	
	\begin{multicols}{2}
		\begin{Verbatim}
		Here's an unnumbered list:
		\begin{itemize}
		\item Item one
		\item Item two
		\item Item three
		\end{itemize}
		\end{Verbatim}
		
		\columnbreak	
		
		Here's an unnumbered list:
		\begin{itemize}
			\item Item one
			\item Item two
			\item Item three
		\end{itemize}
	\end{multicols}
	
	\section{Adjusting margins}
	\section{Floats}
	\section{Creating environments}
	\section{Using packages}
	\section{Using math symbols}
	\section{Document types}
	\section{Managing references}
	
	{\LaTeX} offers an easy way to create a list of references and to cite people's work in your text. This is done by creating an auxiliary BiBTeX file, which is basically a {\TeX} file saved with a .bib extension that contains information on all of your references. 
	
	In this section, you will learn how to create a BiBTeX file, how to cite works, how to create a list of references and how to change the formatting of this list. 
	
	\subsection{Creating a bibtex file}
	
	If you want to cite a paper in your text, the first thing you have to do is upload the information about this paper. Fortunately, google can help you with that: go to \href{https://scholar.google.com.br/}{Google Scholar} and look for ``dupas robinson 2013". You should get the screen below.
	
	\begin{figure}[H]
		\centering
		\includegraphics[width=\linewidth]{../img/scholar_cite}
	\end{figure}

	Now, if you click in \texttt{Cite}, you will see the options below, which are different formats used for citation. If you wanted to cite this papers in a word document, you should copy the format you want to display. However, if you copy the BibTeX entry, you can later choose to display your bibliogrpahy list in any of these formats.

	\begin{figure}[H]
		\centering
		\includegraphics[width=.8\linewidth]{../img/scholar_bibtex}
	\end{figure}

	So click on the \texttt{BibTeX} option and you will see the following code. 	You will notice that this entry indicates the type of work (article, book, chapter), the title, authors and other information that will appear in the bibliography. In its first line, it also specifies a label for the paper, \texttt{dupas2013savings}. This is the shortcut you will use to reference this paper whenever you want to cite it.

	\begin{Verbatim}
		@article{dupas2013savings,
		title={Savings constraints and microenterprise development: Evidence 
		from a field experiment in Kenya},
		author={Dupas, Pascaline and Robinson, Jonathan},
		journal={American Economic Journal: Applied Economics},
		volume={5},
		number={1},
		pages={163--192},
		year={2013},
		publisher={American Economic Association}
		}
	\end{Verbatim}
	
	Copy this text, go to TeXstudio, and start a new file. Note that this should be a completely blank file, with no preamble, no document class and no document environment. Paste the text you copied to the blank file and save it in the same folder as your .tex document under the name \texttt{Bibliogrpahy.bib}.
	
	Now do the same for a couple of other papers. In this example, we will use \citeauthor*{de2008returns} \citeyearpar{de2008returns} and \citeauthor*{alix2012forest} \citeyearpar{alix2012forest}. Note that you can alter the label used to reference for each entry by just editing the entry's first line in your BibTeX file. Let's do that by labeling \citeauthor*{alix2012forest} \citeyearpar{alix2012forest} \texttt{mexicoPES}:
	
	\begin{Verbatim}[commandchars=+\(\)]
@article{+color(CornflowerBlue)mexicoPES+color(black),
	title={Forest conservation and slippage: Evidence from Mexico’s national payments 
		for ecosystem services program},
	author={Alix-Garcia, Jennifer M and Shapiro, Elizabeth N and Sims, Katharine RE},
	journal={Land Economics},
	volume={88},
	number={4},
	pages={613--638},
	year={2012},
	publisher={University of Wisconsin Press}
}
	\end{Verbatim}
	
	\subsection{Using the natbib package}
	
	\begin{verbatim}
		\usepackage{natbib}
		\bibliographystyle{chicago}
	\end{verbatim}
	
	\begin{verbatim}
		\bibliography{Bibliography}
	\end{verbatim}
	
	\begin{figure}[H]
		\centering
		\includegraphics[width=\linewidth]{../img/Chicago}
	\end{figure}

	\begin{verbatim}
		\usepackage[numbers]{natbib}
		\bibliographystyle{chicago}
	\end{verbatim}
	
	\begin{verbatim}
		\bibliography{Bibliography}
	\end{verbatim}
	
	\begin{figure}[H]
		\centering
		\includegraphics[width=\linewidth]{../img/Chicago_numbers}
	\end{figure}
	
	
	\subsection{Citing papers in your text}
	
	Citation 2 \citet{dupas2013savings} \\
	Citation 2 \citep{dupas2013savings} \\
	
	Citation 1 \citet[p.~215]{dupas2013savings} \\
	Citation 1 \citep[p.~215]{dupas2013savings} \\
	
	
	Citation 1 \citep[see][]{dupas2013savings} \\
	Citation 1 \citep[see][p.~215]{dupas2013savings} \\
	
	Citation 2 \citep{alix2012forest} \\
	Citation 3 \citep*{alix2012forest}
	
	\citeauthor{de2008returns} \\
	\citeauthor*{de2008returns} \\
	\citeyear{de2008returns} \\ 
	\citeyearpar{de2008returns} \\ 
	
	
	\subsection{Other natbib options}

	\begin{verbatim}
	\usepackage[comma]{natbib}
	\bibliographystyle{chicago}
	\end{verbatim}

	\bibliography{Bibliography}
	
\end{document}   