\documentclass[12pts]{report}

\usepackage[margin=1in]{geometry}
\usepackage[usenames, dvipsnames]{color}
\usepackage{setspace}
\usepackage{graphicx}
\doublespacing
\usepackage{fancyvrb}
\usepackage{varwidth}
\usepackage{verbatim}
\usepackage{multicol}

\setlength{\columnseprule}{1pt}



\newenvironment{centerverbatim}{%
	\par
	\centering
	\varwidth{\linewidth}%
	\verbatim
}{%
	\endverbatim
	\endvarwidth
	\par
}

% Title Page
\title{DIME DYNAMIC DOCUMENTATION TRAINING }
\author{Luiza Andrade \& Mrijan Rimal} 
\date{\today}


\makeatother


\begin{document}
	

\makeatletter
\begin{titlepage}
	\begin{center}
		\includegraphics[width=0.3\linewidth]{i2i.png}\\[10ex]
		{\LARGE \bfseries  \@title }\\[2ex] 
		{\Large  \@author}\\[20ex] 
		{\large \@date}
	\end{center}
\end{titlepage}
\makeatother

\section*{Introduction}
LaTeX is a very useful tool to display descriptive statistics and analysis results. It allows us to create a document once and every time a do-file is run, the tables are automatically updated in our LaTeX document. \\

To complete these tasks, you should use the dataset and template provided on the DIME dynamic documentation folder. \\

\textbf{Exercise Objectives:}
\begin{enumerate}
	\item Learn how to export tables to \LaTeX.
	\item Learn how to export graphs.
	\item Learn how to present tables in a \LaTeX document
\end{enumerate}

\section*{Exercise 1. Exporting Tables}
This exercise only focuses on export tables in \LaTeX \space format from Stata. Exercise 3 below will show how you import the LaTeX outputs into a single \LaTeX \space file and export it to a PDF. As you'll see, this first part of the exercise is very similar to what you have already been doing when you export tables to excel or word: the only change is that now you'll save them to .tex format. \\

\textbf{Task 1:}  Export a balance table to \LaTeX \space using iebaltab (install by typing ssc install ietoolkit in Stata). Test balance of average annual population growth, life expectancy at birth and GNP per capita across treatment and control groups, using region fixed effects and clustering by region. To save the table to \LaTeX, you'll use the \textit{savetex} option. You'll also add the caption and label to your table using stata. 
\begin{center}
	\colorbox{BurntOrange}{\emph{Tip:} look at \textit{iebaltab} help for more detailed information.}
\end{center}


\textbf{Task 2:}  Export a table with frequency and share of region variable. 
\begin{center}
	\colorbox{BurntOrange}{\emph{Tip:} esttab command has a \LaTeX\space option}
\end{center}

\textbf{Task 3:} Export a regression table to \LaTeX. Run a treatment effect regression on life expectancy at birth controlling for GNP per capita. Display results without fixed effects on column (1) and results with fixed effects on column (2). You can use \textit{estadd} and \textit{scalar} options in \textit{esttab} to add a line for fixed effects.

\section*{Exercise 2: Exporting Graphs}
Similar to exercise 1, this exercise only exports a graph in a format that can be read in a \LaTeX\space editor. \\

\textbf{Task 1:}  Create a twoway kernel density plot of life expectancy at birth across treatment groups and export it to .png or .pdf format. To do this, you'll create the graph as usual, but instead of saving it in Stata format, you will export it using the \textit{graph export} option. 

\textbf{Task 2:}  Use iegraph to export treatment effect graphs into .png format.

\section*{Exercise 3 : Importing Tables into \LaTeX}
Use the DIME dynamic documentation template to create a PDF with the tables and images you exported.

%Add graphs as section 4.
\subsection*{Basic Table}

\textbf{Task:} Import a table you have exported from Stata into this document.
\begin{centerverbatim}
\input{/Raw/balance_table.tex}
\end{centerverbatim}
		
	

\subsection*{Add a Caption}
To add a caption, you have to start a new environment i.e. begin and close a table environment. 
\begin{multicols}{2}
	[Depending on where you put the caption command, the label comes above or below the table.]
	\begin{centerverbatim}
		\begin{table}[H]
			\caption{Add table title}
			\input{/Raw/categorical.tex}
		\end{table}
	\end{centerverbatim}
\columnbreak
	\begin{centerverbatim}
		\begin{table}[H]
			\input{/Raw/categorical.tex}
			\caption{Add table title}
		\end{table}
	\end{centerverbatim}
\end{multicols}


\subsection*{Centering}
\textbf{Task:} Now we want the tables to be centered. This can be done by:

	\begin{centerverbatim}
		\begin{table}[H]
			\centering
			\caption{Add table title}
			\input{/Raw/categorical.tex}
		\end{table}
	\end{centerverbatim}
\pagebreak
\subsection*{AdjustBox}
AdjustBox adjusts the tables you export from Stata to fit the page of your document.\\ 
\begin{centerverbatim}
	\begin{table}[H]
	\centering
	\caption{Descriptive statistics for categorical variables}
	\begin{adjustbox}{max width=\textwidth}   
	\subimport{Raw/}{categorical}
	\end{adjustbox}
	\end{table}
\end{centerverbatim}

\section*{Exercise 4: Importing Graphs into \LaTeX}

\includegraphics{Raw/iegraph.png}

\section*{Exercise 5: Update \LaTeX\space file after making changes to the data}
Change the treatment status of individuals in your dataset through a new randomization process. Rerun the codes for your tables and graphs with the same export path. Update \LaTeX file to view changes.

\section*{Exercise 6: (CHALLENGE) Using a do-file to edit a .tex file after exporting it}

This exercise requires more familiarity with \LaTeX \space than the previous. Don't worry if you can't complete it. \\

\textbf{Task 1:} Run the initial code for exercise 6 in the training do-file. This will create a table with sample sizes for control and treatment groups across regions and in the whole sample. Add this table to the \TeX\space file you created in the previous exercises. How does that look?

\textbf{Task 2:} Open the fragmented .tex file created by Stata. Can you identify the source of the extra spacing?

\textbf{Task 3:} Use \textit{filefilter} command in Stata to filter out the lines or characters in the fragmented file that create the extra spacing. Import the new .tex file and check how it looks.

\textbf{Task 4:} Repeat task 3 if necessary.

\end{document}          

